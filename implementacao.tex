
\section{Desenvolvimento}
\setlength{\parindent}{2cm}

O projeto se divide em três vertentes principais que são elas: o gerador de clocks, sensor velocidade e o projeto semáforo que serão descritas de maneira detalhada a seguir.

\subsection{Gerador de clocks}
\setlength{\parindent}{2cm}

Nesta etapa usamos comandos sequências para estabelecer todos os clocks que necessitamos no sistema e para tando criamos as seguintes variáveis:

\begin{itemize}
\item clk-maquina - Será o clock automático de frequência de 27GHz 

\item temp1, temp0  - Responsável por definir a temporização especifica para cada clock

temp1 =0 e temp0 = 0 clock da temporização A (Sinal acai e guaraná)

temp1 =0 e temp0 = 1 clock da temporização B (sinal acai e guaraná) 

temp1 =1 e temp0 = 0 clock da temporização C (Sinal acai e guaraná)

\item clk-saida-gua,clk-saida-acai - Após toda logica sequencial for executada isso resultara em dois clocks sendo este definidos de acordo com a temporização estabelecida.

\end{itemize}

\setlength{\parindent}{2cm}

obs: Para não haver poluimento visual todo o código se encontrara em anexo

\subsection{Projeto Semáforo}

\setlength{\parindent}{2cm}

Nesta etapa desenvolvemos o sistema de semáforos tendo e para isto se faz necessário o uso de todos os clocks as temporizações para que com elas podemos obter as equações lógicas para cada luz do semáforo, tendo para isso o uso das seguintes variáveis.

\begin{itemize}
\item  clk-maquina - Será o clock automático de frequência de 27GHz 
\item guarana-vermelho,guarana-amarelo,guarana-verde - Luzes do sinal guarana
\item temp1, temp0  - Temporização de cada sinal 
\item acai-vermelho,acai-amarelo,acai-verde - Luzes do sinal acai
\end{itemize}

\setlength{\parindent}{2cm}
Seguindo com o raciocínio usando na implementação do projeto usamos o item Gerador de clock anteriormente como componente e através da funcionalidade portmap, para passar todos os parâmetros necessários para obter todos os clocks para as três temporizações requisitadas.

Para entender o funcionamento do sistema de forma obsoleta vamos analisar a seguir a simulação no wawerform de todos os sinais nas três temporizações que o trabalho estabelece.


Para Temporização A

\vskip3ex
\begin{figure}
\centering
\includegraphics[width=0.99\columnwidth]{w.eps}
\end{figure}
\vskip3ex

Para Temporização B 

\vskip3ex
\begin{figure}
\centering
\includegraphics[width=0.99\columnwidth]{w.eps}
\end{figure}
\vskip3ex

Para Temporização C 

\vskip3ex
\begin{figure}
\centering
\includegraphics[width=0.99\columnwidth]{w.eps}
\end{figure}
\vskip3ex


\subsection{Sensor de Velocidade}

\setlength{\parindent}{2cm}

\section{Resultados Conclusivos}

\setlength{\parindent}{2cm}

Todos os códigos implementados poderão bem como o poster poderão ser encontrados no link a seguir:
 \href{https://github.com/mayaracalixta/Poster}{Git Hub}
