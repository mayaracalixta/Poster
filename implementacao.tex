
\section{Desenvolvimento}
\setlength{\parindent}{2cm}

O projeto se divide em três vertentes principais que são elas o gerador de clocks, sensor velocidade e o projeto semáforo que serão descritas de maneira detalhada a seguir.

\subsection{Gerador de clocks}
\setlength{\parindent}{2cm}

Nesta etapa usamos comandos sequências para estabelecer todos os clocks que necessitamos no sistema e para tando criamos as seguintes variáveis:

\begin{itemize}
\item clk-maquina - Será acionamento manualmente através de seu pino especifico 
\item temp1, temp0  - Responsável por definir a temporização especifica para cada clock, sendo temp1 =0 e temp0 = 0 clock da temporização A (Sinal acai ou  guaraná), temp1 =0 e temp0 = 1 clock da temporização B (sinal acai ou guaraná, temp1 =1 e temp0 = 0 clock da temporização C (Sinal acai ou guaraná) .
\item clk-saida-gua,clk-saida-acai - Após toda logica sequencial for executada isso resultara em dois clocks sendo este definidos de acordo com a temporização estabelecida.
\end{itemize}

\setlength{\parindent}{2cm}

obs: Para não haver poluimento visual todo o código se encontrara em anexo

\subsection{Projeto Semafóro}

\subsection{Sensor de Velocidade}

\setlength{\parindent}{2cm}

Da mesma forma feita nos quesitos anteriores usamos uma estrutura de comandos sequenciais para a obtenção dos resultados.
    
\begin{itemize}
\item clk-maquina -  O clock responsável para definir o funcionamento do percurso do veiculo em questão.  
\item botao1,botao2 - Como a velocidade de um veiculo se limita quando este no sinal amarelo então quando botao1 = 1 e botao2 = 0 contasse quando vezes esse veiculo passa e por fim verificasse se já chegou no limite.
\item velocidade-ok,velocidade-bad - A saída será caracteriza de acordo com o limite especificado anteriormente, se este atingir o limite velocidade-ok = 0 e velocidade-bad = 1 e o contrario caso o limite não for excedido.  
\end{itemize}

\vskip3ex
\begin{figure}
\centering
\includegraphics[width=0.99\columnwidth]{w.eps}
\end{figure}
\vskip3ex


\section{Resultados Conclusivos}

\setlength{\parindent}{2cm}

Todos os códigos implementados poderão bem como o poster poderão ser encontrados no link a seguir:
 \href{https://github.com/mayaracalixta/Poster}{Git Hub}
