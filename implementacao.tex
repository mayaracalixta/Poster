
\section{Metodologia}
\setlength{\parindent}{2cm}

O projeto se divide em três vertentes principais que são elas o gerador de clocks, sensor velocidade e o projeto semáforo que serão descritas de maneira detalhada a seguir.

\subsection{Gerador de clocks}
\setlength{\parindent}{2cm}

Nesta etapa usamos comandos sequências para estabelecer todos os clocks que necessitamos no sistema e para tando criamos as seguintes variáveis:

\begin{itemize}
\item clk-maquina - Será acionamento manualmente através de seu pino especifico 
\item temp1, temp0  - Responsável por definir a temporização especifica para cada clock, sendo temp1 =0 e temp0 = 0 clock da temporização A (Sinal acai ou  guaraná), temp1 =0 e temp0 = 1 clock da temporização B (sinal acai ou guaraná, temp1 =1 e temp0 = 0 clock da temporização C (Sinal acai ou guaraná) .
\item clk-saida-gua,clk-saida-acai - Após toda logica sequencial for executada isso resultara em dois clocks sendo este definidos de acordo com a temporização estabelecida.
\end{itemize}

\setlength{\parindent}{2cm}

obs: Para não haver poluimento visual todo o código se encontrara em anexo

\subsection{Projeto Semafóro}
\subsection{Sensor de Velocidade}

\vskip3ex
\begin{figure}
\centering
\includegraphics[width=0.99\columnwidth]{w.eps}
\end{figure}
\vskip3ex



\setlength{\parindent}{2cm}
O radar de velocidade é composto por dois sensores de pressão localizados na via e que estão distantes 30 metros um do outro. O cálculo da velocidade do veículo é baseado no intervalo de tempo entre as ativações dos dois sensores. Se a velocidade do veículo for superior a 36 quilômetros por hora, o motorista deverá ser processado.

\section{Implementação}

O aluno deverá implementar o sistema de semáforos em FPGA, utilizando os seguintes pinos:
\begin{itemize}
    \justifying
    \item LEDR[0], LEDR[1] e LEDR[2] - Luzes vermelha, amarela e verde do semáforo da Av. Açaí respectivamente.
    \item LEDG[6], LEDG[7] e LEDG[8] - Luzes vermelha, amarela e verde do semáforo da R. Guaraná respectivamente.
    \item SW[0] e SW[1] - Chaves que selecionam os modos: "00" para temporização A, "01" para temporização B e "10" para temporização C.
    \item KEY[0] e KEY[1] -  Sensores de pressão (ativados no '0').
    \item LEDG[0] - Aviso de veículo com velocidade adequada.
    \item LEDR[17] - Aviso de veículo com velocidade inadequada.
\end{itemize}

\section{Apresentação de Resultados}

\setlength{\parindent}{2cm}
Toda a implementação do projeto utilizará apenas os conceitos apresentados em sala de aula, tendo uma pontuação reduzida significativamente caso isso seja violado.

Os resultados deverão ser apresentados em formato de pôster como este disponível em: \href{https://github.com/kenreurisondca/Poster-Circuitos-Digitais}{Pôster Modelo}. Contendo:
\begin{itemize}
    \item Descrição da metodologia;
    \item Diagramas \textit{RTL}, com comentários;
    \item Simulações \textit{waveform} que descrevem o funcionamento dos semáforos e do radar.
\end{itemize}
\vskip2ex

O prazo para a entrega do pôster com os códigos fontes do projeto compactados será \textbf{01 de Novembro} (não precisam imprimir) via SIGAA. E a apresentação do grupo ocorrerá no dia \textbf{02 de Novembro} em sala de aula.